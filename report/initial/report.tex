\documentclass{article}

% if you need to pass options to natbib, use, e.g.:
%     \PassOptionsToPackage{numbers, compress}{natbib}
% before loading neurips_2019

% ready for submission
% \usepackage{neurips_2019}

% to compile a preprint version, e.g., for submission to arXiv, add add the
% [preprint] option:
%     \usepackage[preprint]{neurips_2019}

% to compile a camera-ready version, add the [final] option, e.g.:
\usepackage[final]{neurips_2019}

% to avoid loading the natbib package, add option nonatbib:
%     \usepackage[nonatbib]{neurips_2019}

\usepackage[utf8]{inputenc} % allow utf-8 input
\usepackage[T1]{fontenc}    % use 8-bit T1 fonts
\usepackage{hyperref}       % hyperlinks
\usepackage{xurl}            % simple URL typesetting
\usepackage{booktabs}       % professional-quality tables
\usepackage{amsfonts}       % blackboard math symbols
\usepackage{nicefrac}       % compact symbols for 1/2, etc.
\usepackage{microtype}      % microtypography
\usepackage{graphicx}
\usepackage{xcolor}


\title{Raport inițial - Mașinin autonome - detecție și recunoașterea semnelor de circulație}

% The \author macro works with any number of authors. There are two commands
% used to separate the names and addresses of multiple authors: \And and \AND.
%
% Using \And between authors leaves it to LaTeX to determine where to break the
% lines. Using \AND forces a line break at that point. So, if LaTeX puts 3 of 4
% authors names on the first line, and the last on the second line, try using
% \AND instead of \And before the third author name.

\author{%
 ECHIPĂ: RMA02
 \AND
 Mițca Dumitru - m1 \\
 Grupa 1311A
 \AND
 Mihalache Mihai - m2 \\
 Grupa 1311A
}

\begin{document}


\noindent\begin{minipage}{0.1\textwidth}% adapt widths of minipages to your needs
\includegraphics[width=1.1cm]{imagini/logo_AC.png}
\end{minipage}%
\hfill%
\begin{minipage}{1\textwidth}\raggedright
Universitatea Tehnică "Gheorghe Asachi" din Iași\\
Facultatea de Automatică și Calculatoare\\
Prelucrarea Imaginilor - Proiect
\end{minipage}
% \end{}

\maketitle

\section{Descrierea temei}

% - Descrierea propunerii de proiect: tema aleasă, relevanta proiectului, scopul, obiective SMART (specific, measurable, attainable, relevant,  time based), prezentarea caracterului inovativ, cerințe funcționale, provocari tehnologice

The theme for our project is computer detection and recognition of road signs, a topic that becomes
more relevant every day as the idea of autonomous cars becomes more popular and more work is put
towards making them reality.

Our goal is by the end of the project is to have a program that correctly detects road signs in an image,
and to also output actions based on what's there.

SMART objectives:
\begin{itemize}
  \item detect all road signs in a picture; this can be measured by manually counting the road signs in the picture
  and comparing that against the program output; this is strictly required for our project; this can be achieved until
  meeting 4
  \item identify a road sign in a picture; it is measurable by checking how many signs it identifies correctly in X
  images of signs; this is strictly required for our project; this will be achieved until meeting 6
  \item output actions based on what there is in the image; it is measurably by checking against what a real driver
  would do; this is optional but a nice-to-have; this would be achieved until the end of the semester
\end{itemize}

TODO: caracter inovativ
idei:
1. dataset mic
2. preprocesare algoritmica + segmentare algoritmica
3. comenzi sugerate (incetinesti la atentie copii, opresti la stop, continui la prioritate, ar trb pus pe ceva masina dar de unde, semnalizezi la obligatoriu la dreapta/stanga)
% 4. Folosire a unei camere depth pt a estima distanta de la masina la semn

Functional requirements:
\begin{itemize}
  \item when given an image of a road that contains road signs, our program should correctly detect road signs
  \item when given an image of a road that does not contain road signs, our program should not detect road signs
  \item our program should not misclassify one road sign as another, i.e. a stop sign should not be classified as
  a "you have right of way" sign
  \item our program should not misclassify non-road-sign objects as road signs
\end{itemize}

The technical challenges we will face are: the need for powerful hardware, the need for our program to cope
with bad weather conditions. We do not believe finding a dataset of pictures with road signs to be particularly
difficult, but finding one specific to Romania might be harder, however, a general European dataset would
quite useful

% - Descrierea succintă a rezultatului final al proiectului cu detalierea aplicabilității în industrie și potențiali utilizatori (nevoile identificate în domeniu, analiza cererii pentru rezultatele proiectului, potențiali consumatori, competitori pe piață etc.)

Our project's end product will be a program that takes an image of a road as input and returns the action(s) an
autonomous vehicle should take based on the road signs in it. This program would be useful in the automotive
industry as the development of autonomous vehicles has become more and more popular in recent times.

Our potential clients would be automative companies that have shown interest in developing autonomous vehicles.

\section{Modalitatea de lucru propusă}
%includeți aici link-ul repository-ului de pe GIT
% - modalitatea de lucru propusă (task-urile planificate și alocarea acestora pe membrii echipei)

% \color{red}

% - în planificarea realizată trebuie să luați în considerare următoarele tipuri de activități ce intervin în dezvoltarea proiectului: documentare, implementare, testare, raportare

% \color{black}

\textbf{Identificarea și alocarea task-urilor}

\begin{center}
\begin{tabular}{ |c|c|c| }
 \hline
 \textbf{Task ID} & \textbf{Descriere task} & \textbf{Membru echipă} \\
  \hline
 task1 & Research existing state of the art models & m2 \\ % like YOLO5, YOLO10
  \hline
 task2 & Research libraries and frameworks to be used in the project & m1, m2 \\
 \hline
 task3 & Identify a dataset we can use & m2 \\
 \hline
 task4 & Preprocessing the image & m1 \\
 \hline
 task5 & Segmenting the image into road signs & m1 \\
 \hline
 task6 & Writing the intermediary documentation & m1 \\
 \hline
 task7 & Implementing the classification algorithm (ML/AI/DL) & m2 \\
 \hline
 task8 & Postprocessing the results (aka identifying the action to be taken) & m1, m2 \\
 \hline
 task9 & Writing the final documentation & m2 \\
 \hline
\end{tabular}
\end{center}

\textbf{Git repository:} \url{https://github.com/VedereArtificiala/prelucrareaimaginilor-proiect-itzyabois}

% \section*{Referințe}

% \medskip

% \small
% %Exemple de referințe,
% \color{red}
% Mai jos aveți câteva referințe date ca exemplu. Ștergeți/comentați aceste referințe și adăugați referințele de care aveți nevoie. În cazul în care nu aveți referințe, ștergeți/comentați toată secțiunea \textbf{Referințe}.

% \color{black}
% [1] Alexander, J.A.\ \& Mozer, M.C.\ (1995) Template-based algorithms for
% connectionist rule extraction. In G.\ Tesauro, D.S.\ Touretzky and T.K.\ Leen
% (eds.), {\it Advances in Neural Information Processing Systems 7},
% pp.\ 609--616. Cambridge, MA: MIT Press.

% [2] Bower, J.M.\ \& Beeman, D.\ (1995) {\it The Book of GENESIS: Exploring
%   Realistic Neural Models with the GEneral NEural SImulation System.}  New York:
% TELOS/Springer--Verlag.

% [3] Hasselmo, M.E., Schnell, E.\ \& Barkai, E.\ (1995) Dynamics of learning and
% recall at excitatory recurrent synapses and cholinergic modulation in rat
% hippocampal region CA3. {\it Journal of Neuroscience} {\bf 15}(7):5249-5262.

\end{document}


%pentru a insera un tabel https://www.overleaf.com/learn/latex/tables#Creating_a_simple_table_in_LaTeX
%pentru a insera imagini https://www.overleaf.com/learn/how-to/Including_images_on_Overleaf

%\begin{figure}[htp]
%    \centering
%    \includegraphics[width=4cm]{imagini/logo_AC.png}
%    \caption{Logo AC}
%    \label{fig:logoAC}
%\end{figure}

%pentru a insera liste https://www.overleaf.com/learn/latex/Lists
%ordered 1, 2, 3, ..
%\begin{enumerate}
%  \item The labels consists of sequential numbers.
%  \item The numbers starts at 1 with every call to the enumerate environment.
%\end{enumerate}

%unordered
%\begin{itemize}
%  \item The individual entries are indicated with a black dot, a so-called bullet.
%  \item The text in the entries may be of any length.
%\end{itemize}
