\documentclass{article}

% if you need to pass options to natbib, use, e.g.:
%     \PassOptionsToPackage{numbers, compress}{natbib}
% before loading neurips_2019

% ready for submission
% \usepackage{neurips_2019}

% to compile a preprint version, e.g., for submission to arXiv, add add the
% [preprint] option:
%     \usepackage[preprint]{neurips_2019}

% to compile a camera-ready version, add the [final] option, e.g.:
\usepackage[final]{neurips_2019}

% to avoid loading the natbib package, add option nonatbib:
%     \usepackage[nonatbib]{neurips_2019}

\usepackage[utf8]{inputenc} % allow utf-8 input
\usepackage[T1]{fontenc}    % use 8-bit T1 fonts
\usepackage{hyperref}       % hyperlinks
\usepackage{url}            % simple URL typesetting
\usepackage{booktabs}       % professional-quality tables
\usepackage{amsfonts}       % blackboard math symbols
\usepackage{nicefrac}       % compact symbols for 1/2, etc.
\usepackage{microtype}      % microtypography
\usepackage{graphicx}
\usepackage{xcolor}


\title{Raport inițial - Inserati$\_$aici$\_$titlul$\_$proiectului}

% The \author macro works with any number of authors. There are two commands
% used to separate the names and addresses of multiple authors: \And and \AND.
%
% Using \And between authors leaves it to LaTeX to determine where to break the
% lines. Using \AND forces a line break at that point. So, if LaTeX puts 3 of 4
% authors names on the first line, and the last on the second line, try using
% \AND instead of \And before the third author name.

\author{%
 ECHIPĂ: inserați$\_$aici$\_$codul$\_$echipei
 \AND
 Nume Prenume Student1 \\
 Grupa 1111C
 \AND
 Nume Prenume Student 2 \\
 Grupa 1111C
}

\begin{document}


\noindent\begin{minipage}{0.1\textwidth}% adapt widths of minipages to your needs
\includegraphics[width=1.1cm]{imagini/logo_AC.png}
\end{minipage}%
\hfill%
\begin{minipage}{1\textwidth}\raggedright
Universitatea Tehnică "Gheorghe Asachi" din Iași\\
Facultatea de Automatică și Calculatoare\\
Prelucrarea Imaginilor - Proiect
\end{minipage}
% \end{}

\maketitle

\section{Descrierea temei}

- Descrierea propunerii de proiect: tema aleasă, relevanta proiectului, scopul, obiective SMART (specific, measurable, attainable, relevant,  time based), prezentarea caracterului inovativ, cerințe funcționale, provocari tehnologice

- Descrierea succintă a rezultatului final al proiectului cu detalierea aplicabilității în industrie și potențiali utilizatori (nevoile identificate în domeniu, analiza cererii pentru rezultatele proiectului, potențiali consumatori, competitori pe piață etc.)

\section{Modalitatea de lucru propusă}
%includeți aici link-ul repository-ului de pe GIT
- modalitatea de lucru propusă (task-urile planificate și alocarea acestora pe membrii echipei)

\color{red}

- în planificarea realizată trebuie să luați în considerare următoarele tipuri de activități ce intervin în dezvoltarea proiectului: documentare, implementare, testare, raportare

\color{black}

\textbf{Identificarea și alocarea task-urilor}

\begin{center}
\begin{tabular}{ |c|c|c| } 
 \hline
 \textbf{Task ID} & \textbf{Descriere task} & \textbf{Membru echipă} \\ 
  \hline
 task1 & descriere task1 &  m1, m2 \\ 
  \hline
 task2 & descriere task2 &  m2 \\ 
 \hline
 ... & ... & ... \\ 
 \hline
\end{tabular}
\end{center}

\textbf{Git repository:} http...

\section*{Referințe}

\medskip

\small
%Exemple de referințe, 
\color{red}
Mai jos aveți câteva referințe date ca exemplu. Ștergeți/comentați aceste referințe și adăugați referințele de care aveți nevoie. În cazul în care nu aveți referințe, ștergeți/comentați toată secțiunea \textbf{Referințe}.

\color{black}
[1] Alexander, J.A.\ \& Mozer, M.C.\ (1995) Template-based algorithms for
connectionist rule extraction. In G.\ Tesauro, D.S.\ Touretzky and T.K.\ Leen
(eds.), {\it Advances in Neural Information Processing Systems 7},
pp.\ 609--616. Cambridge, MA: MIT Press.

[2] Bower, J.M.\ \& Beeman, D.\ (1995) {\it The Book of GENESIS: Exploring
  Realistic Neural Models with the GEneral NEural SImulation System.}  New York:
TELOS/Springer--Verlag.

[3] Hasselmo, M.E., Schnell, E.\ \& Barkai, E.\ (1995) Dynamics of learning and
recall at excitatory recurrent synapses and cholinergic modulation in rat
hippocampal region CA3. {\it Journal of Neuroscience} {\bf 15}(7):5249-5262.

\end{document}


%pentru a insera un tabel https://www.overleaf.com/learn/latex/tables#Creating_a_simple_table_in_LaTeX
%pentru a insera imagini https://www.overleaf.com/learn/how-to/Including_images_on_Overleaf

%\begin{figure}[htp]
%    \centering
%    \includegraphics[width=4cm]{imagini/logo_AC.png}
%    \caption{Logo AC}
%    \label{fig:logoAC}
%\end{figure}

%pentru a insera liste https://www.overleaf.com/learn/latex/Lists
%ordered 1, 2, 3, ..
%\begin{enumerate}
%  \item The labels consists of sequential numbers.
%  \item The numbers starts at 1 with every call to the enumerate environment.
%\end{enumerate}

%unordered
%\begin{itemize}
%  \item The individual entries are indicated with a black dot, a so-called bullet.
%  \item The text in the entries may be of any length.
%\end{itemize}